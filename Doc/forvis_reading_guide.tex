% -*- mode: fundamental -*-

% Book: Intro to Computer Architecture, and Empirical approach
% Begun by Nikhil on July 9, 2012

\documentclass[11pt]{article}

% ================================================================
\usepackage[latin1]{inputenc}
\usepackage[T1]{fontenc}
\usepackage{latexsym}
\usepackage{makeidx}
\usepackage{alltt}
\usepackage{verbatim}
\usepackage{fancyvrb}
% \usepackage{moreverb}
\usepackage{ae}
\usepackage{aecompl}

  \usepackage[pdftex,colorlinks=true,bookmarksopen, pdfstartview=FitH,
              linkcolor=blue, citecolor=blue, urlcolor=blue]{hyperref}
  \pdfcompresslevel=9
  \usepackage[pdftex]{graphicx}

% ================================================================

% HORIZONTAL MARGINS
% Left margin, odd pages: 1.00 inch (0.00 + 1)
\setlength{\oddsidemargin}{0.00in}
% Left margin, even pages: 1.00 inch (0.00 + 1)
\setlength{\evensidemargin}{0.00in}
% Text width 6.5 inch (so other margin is 1.00 inch).
\setlength{\textwidth}{6.5in}
% ----------------
% VERTICAL MARGINS
% Top margin 0.5 inch (-0.5 + 1)
\setlength{\topmargin}{-0.5in}
% Head height 0.25 inch (where page headers go)
\setlength{\headheight}{0.25in}
% Head separation 0.25 inch (between header and top line of text)
\setlength{\headsep}{0.25in}
% Text height 9 inch (so bottom margin 1 in)
\setlength{\textheight}{9in}
% ----------------
% PARAGRAPH INDENTATION
\setlength{\parindent}{0in}
% SPACE BETWEEN PARAGRAPHS
\setlength{\parskip}{\medskipamount}
% ----------------
% STRUTS
% HORIZONTAL STRUT.  One argument (width).
\newcommand{\hstrut}[1]{\hspace*{#1}}
% VERTICAL STRUT. Two arguments (offset from baseline, height).
\newcommand{\vstrut}[2]{\rule[#1]{0in}{#2}}
% ----------------
% HORIZONTAL LINE ACROSS PAGE:
\newcommand{\hdivider}{\noindent\mbox{}\hrulefill\mbox{}} 
% ----------------
% EMPTY BOXES OF VARIOUS WIDTHS, FOR INDENTATION
\newcommand{\hm}{\hspace*{1em}}
\newcommand{\hmm}{\hspace*{2em}}
\newcommand{\hmmm}{\hspace*{3em}}
\newcommand{\hmmmm}{\hspace*{4em}}
% ----------------
% VARIOUS CONVENIENT WIDTHS RELATIVE TO THE TEXT WIDTH, FOR BOXES.
\newlength{\hlessmm}
\setlength{\hlessmm}{\textwidth}
\addtolength{\hlessmm}{-2em}

\newlength{\hlessmmmm}
\setlength{\hlessmmmm}{\textwidth}
\addtolength{\hlessmmmm}{-4em}
% ----------------
% ``TIGHTLIST'' ENVIRONMENT (no para space betwee items, small indent)
\newenvironment{tightlist}%
{\begin{list}{$\bullet$}{%
    \setlength{\topsep}{0in}
    \setlength{\partopsep}{0in}
    \setlength{\itemsep}{0in}
    \setlength{\parsep}{0in}
    \setlength{\leftmargin}{1.5em}
    \setlength{\rightmargin}{0in}
    \setlength{\itemindent}{0in}
}
}%
{\end{list}
}
% ----------------
% ITALICISE WORDS
\newcommand{\ie}{\emph{i.e.,}}
\newcommand{\eg}{\emph{e.g.,}}
\newcommand{\Eg}{\emph{E.g.,}}
\newcommand{\etc}{\emph{etc.}}
\newcommand{\via}{\emph{via}}
\newcommand{\vs}{\emph{vs.}}
% ----------------
% CODE FONT (e.g. {\cf x := 0}).
\newcommand{\cf}{\footnotesize\tt}
% ----------------
% KEYWORDS
\newcommand{\kw}[1]{{\bf #1}}

% ----------------------------------------------------------------
% ----------------------------------------------------------------
% HERE BEGINS THE DOCUMENT

\newcommand{\copyrightnotice}{\copyright 2018 R.S.Nikhil; All Rights Reserved}

% ================================================================

\begin{document}

% ----------------------------------------------------------------

\pagestyle{empty}

\begin{center}

\vspace*{1.5in}

{\LARGE\bf A Reading Guide to} \\
{\LARGE\bf FORVIS: A Formal RISC-V ISA Specification}

\vspace{2cm}

{\Large \emph{Rishiyur S. Nikhil}} \\

Bluespec, Inc.


\vspace{0.5in}

\copyright{} 2018 R.S.Nikhil

\vspace{1in}

Revision: \today

\end{center}

% ****************************************************************
% PREFACE AND ACKNOWLEDGEMENTS

\newpage

\pagenumbering{roman}

% ================================================================
% Abbreviations and links

\subsection*{Abbreviations, acronyms and terminology and links}

\begin{tabular}{|l|p{4.5in}|}
\hline
CSR   & Control and Status Register \\
\hline
FPR   & Floating Point Register \\
\hline
GPR   & General Purpose Register \\
\hline
Hart  & Hardware Thread.  Not to be confused with software threads
         such as POSIX threads, ``pthreads'', and processes.
	 A hart has, in hardware, its own PC and fetch unit,
	 and can work concurrently with other harts \\
\hline
ISA   & Instruction Set Architecture \\
\hline
PC    & Program Counter \\
\hline
RVWMO & RISC-V Weak Memory Ordering (default memory model) \\
\hline
Sv32  & Virtual Memory System in RV32 systems \\
\hline
Sv39  & Virtual Memory System in RV64 systems \\
\hline
Sv48  & Optional additional Virtual Memory System in RV64 systems \\
\hline
WMM  & Weak Memory Model \\
\hline
\end{tabular}

\vspace*{1cm}

For more information on terminology and concepts, and information on RISC-V, we recommend these fine books:

\begin{itemize}
\item
``The RISC-V Reader: An Open Architecture Atlas'', by Patterson and Waterman~\cite{Patterson2017b}

\item
``Computer Architecture: A Quantitative Approach'', by Hennessy and Patterson~\cite{Hennessy2017}

\item
``Computer Organization and Design: The Hardware/Software Interface'' (RISC-V Edition) by
     Patterson and Hennessy~\cite{Patterson2017a}
\end{itemize}

and the RISC-V Foundation web site: \verb|https://riscv.org|

% ----------------------------------------------------------------

\subsection*{Acknowledgments}

Thanks to the original creators of RISC-V for making all this possible in the first place.

Thanks to Bluespec, Inc. for supporting this work.

Thanks to the RISC-V Foundation for constituting the ISA Formal
Specification Technical Group.

Thanks to the members of the RISC-V Foundation's ISA Formal
Specification Technical Group with whom we have had many wonderful
discussions on a weekly basis that have inspired and clarified this
work.

% ****************************************************************
% TABLE OF CONTENTS

\newpage

\pagestyle{myheadings}

\markboth{CONTENTS}{}

{\small

\tableofcontents

}

\pagenumbering{arabic}

% ****************************************************************

\newpage

\begin{center}

\vspace*{4in}

{\Large\emph{This page is intentionally blank.}}

\vspace*{1.5in}

\includegraphics[width=2in]{Figs/MagrittePipe.jpg}

{\small ``This is not a pipe''. Ren� Magritte, 1929.

{https://en.wikipedia.org/wiki/The\_Treachery\_of\_Images}\footnote{
Image from Wikipedia and used with same ``fair use'' rationale:
 {https://en.wikipedia.org/wiki/File:MagrittePipe.jpg}}}

\end{center}


% ****************************************************************

\newpage

\section{Introduction}

\markboth{Introduction}{\copyrightnotice}

\setcounter{page}{1}
% \renewcommand{\thepage}{\arabic{page}}

\label{sec_intro}

% ================================================================

This document is in the style of ``literate programming''
(Knuth~1984~\cite{Knuth1984}).  The Forvis spec is actually a
collection of program source files written in the well-known
functional programming language Haskell~\cite{PeytonJones2003}, and
this document is just a reading guide.  All the code fragments herein
are automatically extracted from the actual Haskell code.

In particular: you should not try to read this document on its own;
you should have the code in front of you and refer to this document on
the side for clarification and commentary.

Forvis spec code is meant to be readable by people who may be totally
unfamiliar with Haskell and who may have no interest in learning
Haskell.  It uses a \emph{very} small, extremely simple subset of
Haskell\footnote{
We believe that the Haskell used here is simple enough that only minor
syntactic transformation would be needed to render it into some other
functional language such as SML, OCaml, or Scheme.}
(just simple types, function definition and function
application) and none of the features that may be even slightly
unfamiliar to the audience (no Currying/partial-application, no
lambda-expressions, no laziness, no typeclasses, no monads, etc.)  For
those without prior exposure to Haskell, this document explains the
minimal Haskell notation necessary to read the Forvis spec code.

Using extremely simple Haskell will also make it easier for authors of
new ISA extensions to extend these specs to cover their ISA
extensions, even if they are unfamiliar with Haskell.

Using extremely simple Haskell will also make it easy to parse and
connect to other tools, such as proof assistants, theorem provers, and
so on (including the alternate ``concurrent'' interpreter described at
the end of the next section).

% ================================================================

\subsection{About Forvis}

Forvis is a formal specification of the RISC-V Instruction Set
Architecture, expressed in the Haskell functional programming
language.  By ``formal'' we mean that it is written in a formal
language (Haskell), thereby reducing the possibility of ambiguity,
inconsistency and incompleteness that comes with specifications
written in a natural language such as English.

We chose Haskell because it is a pure functional language, with no
side effects.  ISA specs are sometimes hard to read because hidden
state, and their updates by side-effect are hard to keep track of; in
our Haskell code, all state is visible and all updates can be seen
explicitly as recomputation of state.

In addition to precision and completeness, Forvis also has these goals:
\begin{itemize}

\item {\bf Readability:} This spec should be readable by people who
may be completely unfamiliar with Haskell or other formal
specification languages.  Examples of our target audience:

  % ----------------
  \begin{tightlist}
   \item Compiler writers targeting RISC-V, as a reference explaining the instructions they generate.

   \item RISC-V CPU hardware designers, as a refernce explaining the instructions interpreted by their designs.

   \item Students studying RISC-V.

   \item Designers of new RISC-V ISA extensions, who may want to
   extend these specs to include their extensions.

   \item Users of formal methods, who wish to prove properties
   (especially correctness) of compilers and hardware designs.

  \end{tightlist}
  % ----------------

\item {\bf Modularity:} RISC-V is one of the most modular ISAs.  It
supports:

  % ----------------
  \begin{tightlist}
   \item A couple of base ISAs: RV32 (32-bit) and RV64 (64-bit) (an RV128 base is under development)

   \item Numerous extensions, such as M (Integer Multiply/Divide), A
    (Atomic Memory Ops), F (single precision floating point), D
    (double precision floating point), C (compressed 16b insructions), E (embedded).

   \item An optional Privilege Architecture, with M (machine) and
    optional S (supervisor) and U (user) privilege levels.

   \item Implementation options, such as whether misaligned memory
   accesses are handled or cause a trap, whether interrupt delegation
   is supported or not, etc.

  \end{tightlist}
  % ----------------

  Implementations can combine these flexibly in a 'mix-and-match'
  manner.  Some of these options can coexist in a single
  implementation, and some may be dynamically switched on and off.
  Forvis tries to capture all these possibilities.

\item {\bf Concurrency and non-determinism:} RISC-V, like most modern
ISAs, has opportunities for concurrency and legal non-determinism.
For example, even in a single hart (hardware thread), it is expected
that most implementations will have pipelined (concurrent) fetch and
execute units, and that the instructions returned by the fetch unit
may be unpredictable after earlier code that writes to instruction
memory, unless mediated by a FENCE.I instruction.  RISC-V has a Weak
Memory Model, so that in a multi-hart system, memory-writes by one
hart may be ``seen'' in a different order by another hart unless
mediated by FENCE and AMO instructions.  In particular, different
implementations, and even different runs of the same program on the
same implementation, may return different results from reading memory
on different runs.

\item {\bf Executabality:} Forvis constitutes an ``operational''
semantics (as opposed to an ``axiomatic'' semantics).  The spec can
actually be executed as a Haskell program, representing a RISC-V
``implementation'', i.e., it can execute RISC-V binaries.  The README
file in the code repository explains how to execute the code.

\end{itemize}

% ----------------

\subsubsection{Extension for concurrent behavior and weak-memory models}

Forvis code, as written, may appear to be sequential (one complete
instruction at a time) and, indeed, when executed as a Haskell program
(i.e., on a Haskell interpreter), that is exactly what you get.

However, taking exactly the same specification code (in particular,
the file \verb|Spec.hs|) and viewing it as a generic functional
language (not Haskell); it can be given an alternate machine state
(not the one in \verb|Machine_State.hs|), and it can be executed on an
alternate interpreter (not a Haskell interpreter) that demonstrates
all kinds of concurrencies (e.g., due to pipelining, different kinds
of speculation, and more) and non-deterministic interaction with weak
memory models.  We believe it can describe the complete range of
concurrent behaviors seen in actual implementations (and more
concurrent behaviors not seen in practical implementations).

Describing this alternate interpretation is planned as a follow-up
document.  We have a general idea of how this concurrent interpreter
works but are still working out the details.  The concurrency is not
exposed in the spec text, but is implicit in the data flow.  The
central ideas come from ``implicit dataflow'' computation (cf.
``Implicit Parallel Programming in \emph{pH}''\cite{Nikhil2000a}).

% ================================================================

\subsection{How to read the spec code}

\markboth{How to read}{\copyrightnotice}

\label{sec_how_to_read}

As mentioned earlier, the Forvis spec is Haskell source code.  This
document is just a reading guide, and contains code fragments
automatically extracted from the actual source code.  This document is
not meant to be read on its own, but as a reference for clarification
and commentary while you are reading the actual code.

For all readers, whether familiar with Haskell or not, this guide will
help you navigate the source code; reading the code and files in the
presented order may help you absorb the code most quickly.

Readers familiar with Haskell can skip the following sub-section.

% ----------------------------------------------------------------

\subsubsection{Basic Haskell concepts and notation}

Haskell is a pure functional language: everything is expressed as pure
mathematical functions from arguments to results, and composition of
functions.  There is no sequencing, and no concept of updatable
variables (traditional ``assignment statement'')

Each Haskell file is a Haskell module and has the form:

\hmmmm \
\begin{minipage}[t]{4in}\it
{\tt module} module-name {\tt where} \\
{\tt import} another-module-name \\
... \\
{\tt import} another-module-name \\
... \\
constant-or-function-or-type-definition \\
... \\
constant-or-function-or-type-definition \\
...
\end{minipage}

Comments begin with ``\verb|--|'' and extend through the end of the line.

Haskell relies on ``layout'' to convey text structure, i.e.,
indentation instead of brackets and semicolons. A constant definition
looks like this:

\hmmmm \
\begin{minipage}[t]{4in}\it
{\tt foo ::} type \\
{\tt foo = } value-expression
\end{minipage}

A function definition looks like this:

\hmmmm \
\begin{minipage}[t]{4in}\it
{\tt fn ::} arg-type {\tt ->} ... {\tt ->} arg-type {\tt ->} resul-type \\
{\tt fn} arg ... arg {\tt  =} function-body-expression
\end{minipage}

Note: in Haskell, function arguments, both in definitions
and in applications, are typically just juxtaposed and not enclosed in
parentheses and commas, thus: \\
\hspace*{2in} {\tt fn} \emph{arg} ... \emph{arg} \\
instead of: \\
\hspace*{2in} {\tt fn (} \emph{arg}, ..., \emph{arg} {\tt )}

A definition like this:

\hmmm {\tt type Instr = Word32}

just defines a new type \emph{synonym} ({\tt Instr}) for an existing type ({\tt Word32});
this is done just for readability.

A definition like this:

\hmmm {\tt data} \emph{newtype} = ...

defines a new type; these will be explained as we go along.

For readability, large expressions are sometimes deconstructed using
``{\tt let}'' or ``{\tt where}'' expressions to provide meaningful names to
intermediate sub-expressions. For example, instead of: \\
\hmmmm{\tt x + f y z - g a b c} \\
we may write, equivalently: \\
\hmmmm \
\begin{minipage}[t]{4in}\tt
let \\
\hmm tmp1 = f y z \\
\hmm tmp2 = g a b c \\
\hmm result = x + tmp1 + tmp2 \\
in \\
\hmm result
\end{minipage}

or: \\
\hmmmm \
\begin{minipage}[t]{4in}\tt
result \\
where \\
\hmm tmp1 = f y z \\
\hmm tmp2 = g a b c \\
\hmm result = x + tmp1 + tmp2
\end{minipage}

A conditional expression may be written as an if-then-else:
\begin{tabbing}
\hmmm \= \emph{x} = \= {\tt if} \emph{cond-expr1} \\
      \>            \> {\tt then} \emph{expr1} \\
      \>            \> {\tt else} \= {\tt if} \emph{cond-expr2} \\
      \>            \>            \> {\tt then} \emph{expr2} \\
      \>            \>            \> {\tt else} \emph{expr3}
\end{tabbing}
or may be folded into a definition:
\begin{tabbing}
\hmmm \= \emph{x} \= {\tt |} \emph{cond-expr1} \= {\tt =} \emph{expr1} \\
      \>          \> {\tt |} \emph{cond-expr2} \> {\tt =} \emph{expr2} \\
      \>          \> {\tt |} {\tt True}        \> {\tt =} \emph{expr3}
\end{tabbing}

The following table shows certain operators in Haskell and their
counterparts in C where the notations differ.

\begin{tabular}{|c|c|l|}
\hline
Haskell           & C             & \\
\hline
\verb|not x|        & \verb|! x|    & Boolean negation \\
x \verb|/=| y       & x \verb|!=| y & Not-equals operator \\
x \verb|.&.| y      & x \verb|&| y  & Bitwise AND operator \\
x \verb/.|./ y      & x \verb/|/ y  & Bitwise OR operator \\
\verb|complement x| & \verb|~| x    & Bitwise complement \\
\verb|shiftL x n|   & \verb|x << n| & Left shift  \\
\verb|shiftR x n|   & \verb|x >> n| & Right shift (arith if x is signed, logical otherwise) \\
\hline
\end{tabular}

% ****************************************************************

\section{File Arch\_Defs.hs: basic architectural definitions}

\markboth{Basic Architectural Definitions}{\copyrightnotice}

\label{sec_arch_defs}

% ================================================================

\subsection{Base ISA type}

The following defines a data type {\tt RV} with two possible values,
{\tt RV32} and {\tt RV64}.  It is analogous to an ``enum'' declaration
in C, defining a family of constants.  The {\tt deriving} clause says
that Haskell can automatically extend the equality operator {\tt ==}
to work on values of type {\tt RV}, and that Haskell can automatically
extend the {\tt show()} function to work on such values, producing
Strings {\tt "RV32"} and {\tt "RV64"}, respectively.

\input{Extracted/RV.tex}

% ================================================================

\subsection{Instruction Fields}

Below, we define the type {\tt Instr} and {\tt Instr\_C} to be more
readable synonyms for Haskell's {\tt Word32} and {\tt Word16} types
(32-bit and 16-bit unsigned integers).

We use 32-bits here for all instruction fields, even though
in practice they have fewer bits.

\input{Extracted/Instr.tex}

We define a number of help-functions to extract fields from an
instruction.  For example, the function {\tt ifield\_opcode} takes an
instruction as argument and returns the ``bit slice'' from bits 0
through 6 inclusive (equivalent to Verilog's {\tt instr[6:0]}.
Similarly, we have definitions for other fields of interest.

\input{Extracted/Instr_Field_Functions.tex}

RISC-V instructions come in a few standard formats.  For example, the
``B''-type format (for BRANCH instructions) consists of an opcode, a
3-bit function code, two source registers rs1 and rs2, and a 12-bit
immediate value assembled out of various bits in the instruction.  We
define a function \verb|ifields_B_type| that, given an instruction,
returns these fields:

\input{Extracted/ifields_B_type}

Similar functions are defined for the other standard formats.  These
instructions use \verb|bitSlice|, shift functions, bitwise OR
(\verb/.|./) to extract the fields.

% ================================================================

\subsection{Exception Codes}

We define a type for exception codes, and the values of all the standard exception codes:

\input{Extracted/exception_codes_A}

...

\input{Extracted/exception_codes_B}

% ================================================================

\subsection{Memory responses}

We define a type {\tt Mem\_Result} for responses from memory.  This
may be {\tt Mem\_Result\_Ok} (successful), in which case it returns a
value (irrelevant for STORE instructions, but relevant for LOAD,
load-reserved, store-conditional, and AMO ops).  Otherwise it is a
{\tt Mem\_Result\_Err}, in which case it returns an exception code
(such as misalignment error, an access error, or a page fault.)

\input{Extracted/Mem_Result}

When returning a result, we construct expressions like these:
\begin{tabbing}
\hmmm \= {\tt Mem\_Result\_Ok} \hm \= \emph{value-expression} \\
      \> {\tt Mem\_Result\_Err}    \> \emph{exception-value-expression}
\end{tabbing}

When fielding a result, we deconstruct it using a case-expression like this:
\begin{tabbing}
\hmmm \= {\tt case} mem-result {\tt of} \\
      \> \hm \= {\tt Mem\_Result\_Ok v} \hm {\tt ->} \= \emph{use} v \emph{in an expression} \\
      \>     \> {\tt Mem\_Result\_Err ec}   {\tt ->} \> \emph{use} ec \emph{in an expression}
\end{tabbing}


% ================================================================

\subsection{Privilege Levels}

RISC-V defines 3 standard privileve levels: Machine, Supervisor and User:

\input{Extracted/Priv_Level}

% ****************************************************************

\section{File Machine\_State.hs: architectural and machine state}

\markboth{Architectural and Machine State}{\copyrightnotice}

\label{sec_machine_state}

[Reminder: this is for the simple, sequential,
one-instruction-at-a-time interpreter.  The concurrent interpreter has
a substantially different machine state.]

% ================================================================

\subsection{Handling RV32 and RV64 simultaneously}

Although each hardware implementation will typically be either an RV32
system or an RV64 system, the spec encompasses implementations that
can simultaneously support both.  For example, machine-privilege code
may run in RV64 mode while supervisor- and user-privilege code may run
in RV32 mode.  There is also a future RV128 being defined.

In this spec, which covers RV32 and RV64 and their simultaneous use,
we represent everything using 64 bits.  The semantics of each
instruction are defined to be governed by the current RV setting which
is available in the architectural state (specifically, MISA.MXL,
MSTATUS.SXL, MSTATUS.UXL, etc.).  An RV32 setting can render some
instructions illegal, and limits calculations on values to be done
with 32-bit arithmetic.

% ================================================================

\subsection{Machine State}

We define a new type representing a complete ``machine state'', which
is just a record or struct.  The first few fields represent a RISC-V
hart's basic architectural state: a Program Counter, General Purpose
Registers, Control-and-Status Registers, and the current privilege
level at which it is running: This is followed by two fields
representing memory and memory-mapped I/O devices, and finally by
fields that are not semantically relevant and are used just for
redundant information, simulation configuration options, simulation
state, gathered statistics, and so on.

\input{Extracted/Machine_State}

This record-with-fields is a purely internal representation choice in
this module.  Clients of this module only access it via the API
functions that follow.\footnote{Haskell has export-import mechanisms
to enforce this external invisibility of our representation choice,
but we have omitted them to avoid clutter.}

The following function is a constructor that returns a new machine.

\input{Extracted/Machine_State_constructor}

Two typical API functions on the machine state are these, to read and
write the PC.

\input{Extracted/PC_access}

In the API function \verb|mstate_gpr_write|, we ensure that if we are
in RV32 mode, we sign-extend the lower 32 bits of the written value,
before using the ``raw'' \verb|gpr_write| function:

\input{Extracted/mstate_gpr_write}

``Raw'' reads and writes on the GPR file are described in the file
\verb|GPR_File.hs|.

% ****************************************************************

\section{File Spec.hs: the ISA spec}

\markboth{The ISA Spec}{\copyrightnotice}

\label{sec_ISA_spec}

The entire spec is essentially in this one file.  The major sections
are:
\begin{itemize}

\item A function \verb|instr_fetch| expressing instruction-fetch,
    returning a regular 32-bit instruction, or 'C' compressed 16-bit
    instruction, or an instruction-fetch fault.

\item A large number of functions named \verb|spec_|{\it{}OPCODE}
describing the semantics of all RISC-V instructions.

\item A small number of functions name \verb|finish_|{\it{}scheme}
capturing the few common schemes by which instructions finish (write
Rd, increment PC, increment MINSTRET, trap, ...)

\item A function \verb|mstate_upd_on_trap| (which is perhaps the most
intricate) that updates the machine state for a trap.  It computes the
new privilege level, new PC, new MSTATUS, new MEPC/{\linebreak[0]}SEPC/{\linebreak[0]}UEPC, new
MCAUSE/{\linebreak[0]}SCAUSE/{\linebreak[0]}UCAUSE, and new MTVAL/{\linebreak[0]}STVAL/{\linebreak[0]}UTVAL based on whether
it was an interrupt or synchronous trap, the current privilege level,
the MSTATUS register, MIP and MIE registers, MIDELEG and MEDELEG
registers, MTVEC/{\linebreak[0]}STVEC/{\linebreak[0]}UTVEC

\item A function \verb|exec_instr| (and it's counterpart
\verb|exec_instr_C| for 'C' compressed instructions) that uses all the
\verb|spec_|{\it{}OPCODE} to update the machine state by executing one
instruction.

\item A function \verb|take_interrupt_if_any| that checks the machine
state to see if an interrupt is pending and updates the machine state
if so.

\end{itemize}

% ================================================================

\subsection{Instruction fetch}

The start of the code for instruction fetch looks like this:

\input{Extracted/instr_fetch}

The \verb|instr_fetch| function takes the current machine state as
argument, and attempts to read and instruction from memory, returning
a 2-tuple: a \verb|Fetch Result| and the updated machine state.

The \verb|Fetch_Result| can indicate that there was a fault (such as a
memory access fault or page fault) during the attempted memory-read,
or that the instruction is a 16-bit instruction from the 'C' ISA
extension, or that the instructionis a regular 32-bit instruction.  In
all cases, there could have been a change in the machine state, and
hence it returns the updated machine state.

The code that follows the above excerpt first reads 2 bytes from
memory, and checks if it encodes a possible 'C' instruction. If not,
it then reads 2 more bytes from memory and returns it as a full 32-bit
instruction.  Of course, either of these two reads can fault, and this
is the reason we read two bytes at a time: the first read may succeed
with a 'C' instruction, in which case we do not want to encounter a
fault for reading two more bytes which may be unnecessary in the
program flow.

% ================================================================

\subsection{General structure of spec instruction-semantics functions}

The spec is written as a collection of functions, generally one per
major opcode (7 least-significant bits of an instruction).  Each
function has the following structure:

{\small
\begin{Verbatim}[frame=single, numbers=left, commandchars=\\\{\}]
spec_{\it{}OPCODE} :: Machine_State -> Instr -> (Bool, Machine_State) \\
spec_{\it{}OPCODE} mstate  instr =
    -- Instr fields: {\it X}-type
    {\it ... extract instruction fields from standard X format ...}

    -- Decode check
    is_legal = {\it ... check if legal OPCODE and fields ...}

    -- Semantics
    mstate1 = {\it ... instruction-specific semantics based on {\tt mstate} ...}

    mstate2 = {\it ... small family of ``finish'' functions}
  in
    (is_legal, mstate2)
\end{Verbatim}
}

The function takes a machine state and an instruction as arguments,
and returns a 2-tuple: a Boolean and a new machine state.  The Boolean
result indicates whether the instruction is a legal OPCODE instruction
(it's 7 least-significant bits code for OPCODE, and other constraints
on other fields are met, such as must-be-zero).

If the Boolean result is True, the second component of the 2-tuple
result is the updated state due to execution of the instruction.

\subsubsection{Example: the ADD instruction}

The ADD instruction is handled by the following function:

\input{Extracted/spec_ADD_1}
...
\input{Extracted/spec_ADD_2}

It first extracts the fields of an R-type instruction.  Then we define
some booleans like \verb|is_ADD| that check the sub-opcodes of the
instruction.  Then we compute \verb|is_legal| to checks the opcode and
all the sub-opcodes covered by this function.

The next few lines express the semantics of this family of
instructions.  We read the Rs1 and Rs2 registers:

\input{Extracted/spec_ADD_3}

and we compute the value to be stored in the Rd register:

\input{Extracted/spec_ADD_4}

Note that ADD, SUB and SLT treat the register values as signed values,
whereas SLTU treats them as unsigned values. Finally, we invoke the
``finish'' function that writes a value to Rd and increments PC by 4
(and increments MINSTRET), see discussion in
Sec.~\ref{sec_standard_finish_functions}.

\input{Extracted/spec_ADD_5}



% ================================================================

\subsection{Standard ``finish'' functions}

\label{sec_standard_finish_functions}

All instructions ``finish'' in one of a few common ways, and these are
captured as functions.  For example: this function captures the common
finish of all ALU instructions, which write a result value
\verb|rd_val| to the GPR \verb|rd|, increment the PC by 4, and
increment the MINSTRET (instructions retired) counter.

\input{Extracted/finish_rd_and_pc_plus_4}

% ================================================================

\subsection{Interrupts}

\label{sec_interrupts}

\input{Extracted/take_interrupt}

The \verb|take_interrupt_if_any| function can be applied between any
two instruction executions. It uses the function
\verb|fn_interrupt_pending| that examines MSTATUS, MIP, MIE and the
current privilege level to check if there is an interrupt is pending
and the hart is ready to handle it.  If so, it applies
\verb|mstate_upd_on_trap| to update the machine state, which it
returns along with True.  Otherwise, it returns False and the
unchanged machine state.

% ================================================================

\subsection{Sequential (one-instruction-at-a-time)  interpretation}

The sequential interpreter has a machine state \verb|M| as described
in Sec.~\ref{sec_machine_state}, and a list \emph{spec\_fns} of spec
functions as described in the previous section, i.e., each having the
type:

\hmmm {\tt Machine\_State -> Instr -> (Bool, Machine\_State)}

The interpreter performs the following, forever:

\hmm \begin{minipage}[t]{5in}

It uses the memory-access API function \verb|mstate_mem_read| to read
an instruction from \verb|M|.  It then applies each function from
\emph{spec\_fns}, one by one until one of them returns
\verb|(True,M')|, i.e., one of them successfully decodes and executes
the instruction.

\vspace*{1ex}

If all the functions in \emph{spec\_fns} return \verb|(False,...)|,
the interpreter applies the \verb|finish_trap| function to \verb|M|
with the \verb|ILLEGAL_INSTRUCTION| exception code to produce the next
state \verb|M'|.

\end{minipage}

% ****************************************************************

\section{Files GPR\_File.hs (General Purpose Registers and CSR\_File.hs (Control and Status Registers)}

\label{sec_gprs_csrs}

\verb|GPR_File.hs| implements a file of general-purpose registers, and
the API functions \verb|gpr_read| and \verb|gpr_write|; it is simple
enough that we do not discuss it further here.

\verb|CSR_File.hs| implements a file of Control and Status registers,
and the API functions \verb|csr_read| and \verb|csr_write|.  The main
subtlety here is that the distinct CSR addresses refer to ``views'' of
the same underlying register with various restrictions:

\begin{itemize}
\item
\verb|USTATUS| and \verb|SSTATUS| are restricted views of \verb|MSTATUS|

\item
\verb|UIE| and \verb|SIE| are restricted views of \verb|MIE|

\item
\verb|UIP| and \verb|SIP| are restricted views of \verb|MIP|
\end{itemize}


The functions \verb|mstatus_stack_fields| and
\verb|mstatus_upd_stack_fields| encapsulate reading and writing the
``stack'' in the MSTATUS register containing the ``previous
privilege'', ``previous interrupt enable'' and ``interrupt enable''
fields.  This stack is pushed on traps/interrupts, and popped on
URET/SRET/MRET instructions.

The function \verb|fn_interrupt_pending| was mentioned earlier in
Sec.~\ref{sec_interrupts}; it analyzes the MSTATUS, MIP, MIE and
current privilege level to decide whether a machine/supervisor/user
external/software/timer interrupt is pending, and if so, which one.

% ****************************************************************

\newpage

\markboth{BIBLIOGRAPHY}{\copyrightnotice}

\addcontentsline{toc}{section}{Bibliography}

\bibliographystyle{abbrv}
\bibliography{forvis_reading_guide}

% ****************************************************************

\end{document}
