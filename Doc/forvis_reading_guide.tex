% -*- mode: fundamental -*-

\documentclass[11pt]{article}

% ================================================================
\usepackage[latin1]{inputenc}
\usepackage[T1]{fontenc}
\usepackage{latexsym}
\usepackage{makeidx}
\usepackage{alltt}
\usepackage{verbatim}
\usepackage{fancyvrb}
% \usepackage{moreverb}
\usepackage{ae}
\usepackage{aecompl}

  \usepackage[pdftex,colorlinks=true,bookmarksopen, pdfstartview=FitH,
              linkcolor=blue, citecolor=blue, urlcolor=blue]{hyperref}
  \pdfcompresslevel=9
  \usepackage[pdftex]{graphicx}

% ================================================================

% HORIZONTAL MARGINS
% Left margin, odd pages: 1.00 inch (0.00 + 1)
\setlength{\oddsidemargin}{0.00in}
% Left margin, even pages: 1.00 inch (0.00 + 1)
\setlength{\evensidemargin}{0.00in}
% Text width 6.5 inch (so other margin is 1.00 inch).
\setlength{\textwidth}{6.5in}
% ----------------
% VERTICAL MARGINS
% Top margin 0.5 inch (-0.5 + 1)
\setlength{\topmargin}{-0.5in}
% Head height 0.25 inch (where page headers go)
\setlength{\headheight}{0.25in}
% Head separation 0.25 inch (between header and top line of text)
\setlength{\headsep}{0.25in}
% Text height 9 inch (so bottom margin 1 in)
\setlength{\textheight}{9in}
% ----------------
% PARAGRAPH INDENTATION
\setlength{\parindent}{0in}
% SPACE BETWEEN PARAGRAPHS
\setlength{\parskip}{\medskipamount}
% ----------------
% STRUTS
% HORIZONTAL STRUT.  One argument (width).
\newcommand{\hstrut}[1]{\hspace*{#1}}
% VERTICAL STRUT. Two arguments (offset from baseline, height).
\newcommand{\vstrut}[2]{\rule[#1]{0in}{#2}}
% ----------------
% HORIZONTAL LINE ACROSS PAGE:
\newcommand{\hdivider}{\noindent\mbox{}\hrulefill\mbox{}} 
% ----------------
% EMPTY BOXES OF VARIOUS WIDTHS, FOR INDENTATION
\newcommand{\hm}{\hspace*{1em}}
\newcommand{\hmm}{\hspace*{2em}}
\newcommand{\hmmm}{\hspace*{3em}}
\newcommand{\hmmmm}{\hspace*{4em}}
% ----------------
% VARIOUS CONVENIENT WIDTHS RELATIVE TO THE TEXT WIDTH, FOR BOXES.
\newlength{\hlessmm}
\setlength{\hlessmm}{\textwidth}
\addtolength{\hlessmm}{-2em}

\newlength{\hlessmmmm}
\setlength{\hlessmmmm}{\textwidth}
\addtolength{\hlessmmmm}{-4em}
% ----------------
% ``TIGHTLIST'' ENVIRONMENT (no para space betwee items, small indent)
\newenvironment{tightlist}%
{\begin{list}{$\bullet$}{%
    \setlength{\topsep}{0in}
    \setlength{\partopsep}{0in}
    \setlength{\itemsep}{0in}
    \setlength{\parsep}{0in}
    \setlength{\leftmargin}{1.5em}
    \setlength{\rightmargin}{0in}
    \setlength{\itemindent}{0in}
}
}%
{\end{list}
}
% ----------------
% ITALICISE WORDS
\newcommand{\ie}{\emph{i.e.,}}
\newcommand{\eg}{\emph{e.g.,}}
\newcommand{\Eg}{\emph{E.g.,}}
\newcommand{\etc}{\emph{etc.}}
\newcommand{\via}{\emph{via}}
\newcommand{\vs}{\emph{vs.}}
% ----------------
% CODE FONT (e.g. {\cf x := 0}).
\newcommand{\cf}{\footnotesize\tt}
% ----------------
% KEYWORDS
\newcommand{\kw}[1]{{\bf #1}}

% ----------------------------------------------------------------
% ----------------------------------------------------------------
% HERE BEGINS THE DOCUMENT

\newcommand{\copyrightnotice}{\copyright 2018 R.S.Nikhil; All Rights Reserved}

% ================================================================

\begin{document}

% ----------------------------------------------------------------

\pagestyle{empty}

\begin{center}

\vspace*{1.5in}

{\LARGE\bf Forvis: A Formal RISC-V ISA Specification}

\vspace{1cm}

{\large\bf A Reading Guide}

\vspace{2cm}

{\Large \emph{Rishiyur S. Nikhil}} \\

Bluespec, Inc.


\vspace{0.5in}

\copyright{} 2018-2019 R.S.Nikhil

\vspace{1in}

Revision: \today

\end{center}

% ****************************************************************
% PREFACE AND ACKNOWLEDGEMENTS

\newpage

\pagenumbering{roman}

% ================================================================
% Abbreviations and links

\subsection*{Abbreviations, acronyms and terminology and links}

\begin{tabular}{|l|p{4.5in}|}
\hline
CSR   & Control and Status Register \\
\hline
FPR   & Floating Point Register \\
\hline
GPR   & General Purpose Register \\
\hline
Hart  & Hardware Thread.  Not to be confused with software threads
         such as POSIX threads, ``pthreads'', and processes.
	 A hart has, in hardware, its own PC and fetch unit,
	 and can work concurrently with other harts \\
\hline
ISA   & Instruction Set Architecture \\
\hline
PC    & Program Counter \\
\hline
RVWMO & RISC-V Weak Memory Ordering (default memory model) \\
\hline
spec  & Specification \\
\hline
Sv32  & Virtual Memory System in RV32 systems \\
\hline
Sv39  & Virtual Memory System in RV64 systems \\
\hline
Sv48  & Optional additional Virtual Memory System in RV64 systems \\
\hline
WMM  & Weak Memory Model \\
\hline
\end{tabular}

\vspace*{1cm}

For more information on terminology and concepts, and information on RISC-V, we recommend these fine books:

\begin{itemize}
\item
``The RISC-V Reader: An Open Architecture Atlas'', by Patterson and Waterman~\cite{Patterson2017b}

\item
``Computer Architecture: A Quantitative Approach'', by Hennessy and Patterson~\cite{Hennessy2017}

\item
``Computer Organization and Design: The Hardware/Software Interface'' (RISC-V Edition) by
     Patterson and Hennessy~\cite{Patterson2017a}
\end{itemize}

and the RISC-V Foundation web site: \verb|https://riscv.org|

% ----------------------------------------------------------------

\subsection*{Acknowledgments}

Thanks to the original creators of RISC-V for making all this possible in the first place.

Thanks to Bluespec, Inc. for supporting this work.

Thanks to the RISC-V Foundation for constituting the ISA Formal
Specification Technical Group.

Thanks to the members of the RISC-V Foundation's ISA Formal
Specification Technical Group with whom we have had many wonderful
discussions on a weekly basis that have inspired and clarified this
work.

% ****************************************************************
% TABLE OF CONTENTS

\newpage

\pagestyle{myheadings}

\markboth{CONTENTS}{}

{\small

\tableofcontents

}

\pagenumbering{arabic}

% ****************************************************************

\newpage

\begin{center}

\vspace*{4in}

{\Large\emph{This page is intentionally blank.}}

\vspace*{1.5in}

\includegraphics[width=2in]{Figs/MagrittePipe.jpg}

{\small ``This is not a pipe''. Ren� Magritte, 1929.

{https://en.wikipedia.org/wiki/The\_Treachery\_of\_Images}\footnote{
Image from Wikipedia and used with same ``fair use'' rationale:
 {https://en.wikipedia.org/wiki/File:MagrittePipe.jpg}}}

\end{center}


% ****************************************************************

\newpage

\section{Introduction}

\markboth{Introduction}{\copyrightnotice}

\setcounter{page}{1}
% \renewcommand{\thepage}{\arabic{page}}

\label{sec_intro}

% ================================================================

This is a reading guide for Forvis, a formal RISC-V ISA specification
written in ``extremely elementary'' Haskell. It can be executed as a
RISC-V simulator which, in turn, executes RISC-V binaries.

A separate document describes how to build the spec as a simulator and
have it execute RISC-V ELF binary.  Another document describes how to
extend this spec for new ISA extensions.

This is a work-in-progress, one of several similar concurrent efforts
within the ``ISA Formal Specification'' Technical Group constituted by
The RISC-V Foundation ({\tt https://riscv.org}).  We welcome your
feedback, comments and suggestions.\footnote{Forvis, and this
document, are available at: {\tt https://github.com/rsnikhil/RISCV-ISA-Spec}}

Forvis corresponds to these original English-text specs:
\begin{itemize}

\item {\it The RISC-V Instruction Set Manual, Volume I: Unprivileged ISA},
    Andrew Waterman and Krste Asanovic,
    Document Version 20181106-Base-Ratification,
    November 6, 2018.~\cite{Waterman2018_user}

\item {\it The RISC-V Instruction Set Manual, Volume II: Privileged
    Architecture}, 
    Andrew Waterman and Krste Asanovic (eds.),
    Document Version 20181203-Base-Ratification,
    December 3, 2018.~\cite{Waterman2018_priv}:

\end{itemize}

% ================================================================

\subsection{Forvis goals}

Forvis is a formal specification of the RISC-V Instruction Set
Architecture, i.e., it is written in a precise, unambiguous language
(here, Haskell) without regard to hardware implementation
considerations; clarity and precision are paramount concerns.  In
contrast, specs written a natural language such as English are often
prone to ambiguity, inconsistency and incompleteness.  Further, a
formal spec can be parsed and processed automatically, connecting to
other formal analysis and transformation tools.  In addition to
precision and completeness, Forvis also has these goals:

\begin{itemize}

\item {\bf Readability:} This spec should be readable by people who
may be completely unfamiliar with Haskell or other formal
specification languages.  Examples of our target audience:

  % ----------------
  \begin{tightlist}
   \item RISC-V Assembly Language programmers as a reference explaining the instructions they use.

   \item Compiler writers targeting RISC-V, as a reference explaining the instructions they generate.

   \item RISC-V CPU hardware designers, as a refernce explaining the instructions interpreted by their designs.

   \item Students studying RISC-V.

   \item Designers of new RISC-V ISA extensions, who may want to
   extend these specs to include their extensions.

   \item Users of formal methods, who wish to prove properties
   (especially correctness) of compilers and hardware designs.

  \end{tightlist}
  % ----------------

\item {\bf Modularity:} RISC-V is one of the most modular ISAs.  It
supports:

  % ----------------
  \begin{tightlist}
   \item A couple of base ISAs: RV32 (32-bit) and RV64 (64-bit) (an RV128 base is under development)

   \item Numerous extensions, such as M (Integer Multiply/Divide), A
    (Atomic Memory Ops), F (single precision floating point), D
    (double precision floating point), C (compressed 16b insructions), E (embedded).

   \item An optional Privilege Architecture, with M (machine) and
    optional S (supervisor) and U (user) privilege levels.

   \item Implementation options, such as whether misaligned memory
   accesses are handled or cause a trap, whether interrupt delegation
   is supported or not, etc.

  \end{tightlist}
  % ----------------

  Implementations can combine these flexibly in a 'mix-and-match'
  manner.  Some of these options can coexist in a single
  implementation, and some may be dynamically switched on and off.
  Forvis tries to capture all these possibilities.

\item {\bf Concurrency and non-determinism:} RISC-V, like most modern
ISAs, has opportunities for concurrency and legal non-determinism.
For example, even in a single hart (hardware thread), it is expected
that most implementations will have pipelined (concurrent) fetch and
execute units, and that the instructions returned by the fetch unit
may be unpredictable after earlier code that writes to instruction
memory, unless mediated by a FENCE.I instruction.  RISC-V has a Weak
Memory Model, so that in a multi-hart system, memory-writes by one
hart may be ``seen'' in a different order by another hart unless
mediated by FENCE and AMO instructions.  In particular, different
implementations, and even different runs of the same program on the
same implementation, may return different results from reading memory
on different runs.

\item {\bf Executabality:} Forvis constitutes an ``operational''
semantics (as opposed to an ``axiomatic'' semantics).  The spec can
actually be executed as a Haskell program, representing a RISC-V
``implementation'', i.e., it can execute RISC-V binaries.  The README
file in the code repository explains how to execute the code.

\end{itemize}

% ----------------

\subsubsection{Extension for concurrent behavior and weak memory models}

Although it is convenient to directly execute this Haskell code as a
Haskell program, thereby giving us a sequential RISC-V simulator for
free, the code (specifically, the file \verb|Forvis_Spec.hs|) can also
be treated as a generic functional program with an alternate
interpretation (non-Haskell, and changing what we mean by the
``Machine State'' that is an argument to each spec function).

Such an alternate interpreter can demonstrate all kinds of
concurrencies (e.g., due to out-of-order execution, pipelining,
different kinds of speculation, and more) and non-deterministic
interaction with weak memory models.  We believe it can describe the
complete range of concurrent behaviors seen in actual implementations
(and more concurrent behaviors not seen in practical implementations).

Describing this alternate interpretation is planned as a follow-up
document.  We have a general idea of how this concurrent interpreter
works but are still working out the details.  The concurrency is not
exposed in the spec text, but is implicit in the data flow.  The
central ideas come from ``implicit dataflow'' computation (cf.
``Implicit Parallel Programming in \emph{pH}''\cite{Nikhil2000a}).

% ================================================================

\subsection{About the choice of Haskell, and the level of Haskell features used}

We chose to use the well-known programming language
Haskell~\cite{PeytonJones2003} because it is a pure functional
language, with no side effects.  ISA specs are sometimes hard to read
because of hidden state, and their updates by side-effect are hard to
keep track of; in our Haskell code, all state is visible and all
updates can be seen explicitly as recomputation of new state.

Forvis spec code is written in ``extremely elementary'' Haskell so
that it is readable by people who may be totally unfamiliar with
Haskell and who may have no interest in learning Haskell.  It uses a
\emph{very} small, extremely simple subset of Haskell\footnote{ We
believe that the Haskell used here is simple enough that only minor
syntactic transformation would be needed to render it into some other
functional language such as SML, OCaml, or Scheme.}  (just simple
types, function definition and function application) and none of the
features that may be even slightly unfamiliar to the audience (no
Currying/partial-application, lambda-expressions, laziness,
typeclasses, monads, etc.)  For those without prior exposure to
Haskell, this document explains the minimal Haskell notation necessary
to read the Forvis spec code.

Using extremely simple Haskell will also make it easier for authors of
new ISA extensions to extend these specs to cover their ISA
extensions, even if they are unfamiliar with Haskell.

Using extremely simple Haskell will also make it easy to parse and
connect to other tools, such as proof assistants, theorem provers, and
so on (including the alternate ``concurrent'' interpreter described at
the end of the next section).

% ----------------

\subsubsection{The code in this document is real}

This document is produced with a kind of ``literate programming''
process (Knuth~1984~\cite{Knuth1984}).  The Forvis spec \emph{is} the
collection of Haskell source code files, and this document is just a
reading guide.  All the code fragments herein are automatically
extracted from the actual Haskell code during document production.  As
a reading guide, this document is not meant to be read on its own, but
as an accompaniment to perusing the actual source code.

% ================================================================

\subsection{How to read the spec code}

\markboth{How to read}{\copyrightnotice}

\label{sec_how_to_read}

As mentioned earlier, the Forvis spec is Haskell source code.  This
document is just a reading guide, and contains code fragments
automatically extracted from the actual source code.  This document is
not meant to be read on its own, but as a reference for clarification
and commentary while you are reading the actual code.

For all readers, whether familiar with Haskell or not, this guide will
help you navigate the source code; reading the code and files in the
presented order may help you absorb the code most quickly.

Readers familiar with Haskell can skip the following sub-section.

% ----------------------------------------------------------------

\subsubsection{Basic Haskell concepts and notation}

Haskell is a pure functional language: everything is expressed as pure
mathematical functions from arguments to results, and composition of
functions.  There is no sequencing, and no concept of updatable
variables (traditional ``assignment statement'')

Each Haskell file is a Haskell module and has the form:

\hmmmm \
\begin{minipage}[t]{4in}\it
{\tt module} module-name {\tt where} \\
{\tt import} another-module-name \\
... \\
{\tt import} another-module-name \\
... \\
constant-or-function-or-type-definition \\
... \\
constant-or-function-or-type-definition \\
...
\end{minipage}

Comments begin with ``\verb|--|'' and extend through the end of the line.

Haskell relies on ``layout'' to convey text structure, i.e.,
indentation instead of brackets and semicolons. A constant definition
looks like this:

\hmmmm \
\begin{minipage}[t]{4in}\it
{\tt foo = } value-expression {\tt ::} type
\end{minipage}

A function definition looks like this:

\hmmmm \
\begin{minipage}[t]{4in}\it
{\tt fn ::} arg-type {\tt ->} ... {\tt ->} arg-type {\tt ->} resul-type \\
{\tt fn} arg ... arg {\tt  =} function-body-expression
\end{minipage}

Note: in Haskell, function arguments, both in definitions
and in applications, are typically just juxtaposed and not enclosed in
parentheses and commas, thus: \\
\hspace*{2in} {\tt fn} \emph{arg} ... \emph{arg} \\
instead of: \\
\hspace*{2in} {\tt fn (} \emph{arg}, ..., \emph{arg} {\tt )}

A definition like this:

\hmmm {\tt type Instr = Word32}

just defines a new type \emph{synonym} ({\tt Instr}) for an existing type ({\tt Word32});
this is done just for readability.

A definition like this:

\hmmm {\tt data} \emph{newtype} = ...

defines a new type; these will be explained as we go along.

For readability, large expressions are sometimes deconstructed using
``{\tt let}'' expressions to provide meaningful names to intermediate
sub-expressions, define local help-functions, etc. For example,
instead of: \\
\hmmmm{\tt x + f y z - g a b c} \\
we may write, equivalently: \\
\hmmmm \
\begin{minipage}[t]{4in}\tt
let \\
\hmm tmp1 = f y z \\
\hmm tmp2 = g a b c \\
\hmm result = x + tmp1 + tmp2 \\
in \\
\hmm result
\end{minipage}

Conditional expressions may be written using \verb|if-then-else| which can of course be nested:
\begin{tabbing}
\hmmm \= \emph{x} = \= {\tt if} \emph{cond-expr1} \\
      \>            \> {\tt then} \emph{expr1} \\
      \>            \> {\tt else} \= {\tt if} \emph{cond-expr2} \\
      \>            \>            \> {\tt then} \emph{expr2} \\
      \>            \>            \> {\tt else} \emph{expr3}
\end{tabbing}
or using \verb|case| which can also be nested:
\begin{tabbing}
\hmmm \= \emph{x} = \= {\tt case} \emph{cond-expr1} {\tt of}\\
      \>            \> \hmm {\tt True -> } \emph{expr1} \\
      \>            \> \hmm {\tt False ->} \= {\tt case} \emph{cond-expr2} {\tt of}\\
      \>            \>                     \> \hmm {\tt True ->} \emph{expr2} \\
      \>            \>                     \> \hmm {\tt False ->} \emph{expr3}
\end{tabbing}
or may be folded into a definition:
\begin{tabbing}
\hmmm \= \emph{x} \= {\tt |} \emph{cond-expr1} \= {\tt =} \emph{expr1} \\
      \>          \> {\tt |} \emph{cond-expr2} \> {\tt =} \emph{expr2} \\
      \>          \> {\tt |} {\tt True}        \> {\tt =} \emph{expr3}
\end{tabbing}

The following table shows some operators in Haskell and their
counterparts in C, where the notations differ.

\begin{tabular}{|c|c|l|}
\hline
Haskell           & C             & \\
\hline
\verb|not x|        & \verb|! x|    & Boolean negation \\
x \verb|/=| y       & x \verb|!=| y & Not-equals operator \\
x \verb|.&.| y      & x \verb|&| y  & Bitwise AND operator \\
x \verb/.|./ y      & x \verb/|/ y  & Bitwise OR operator \\
\verb|complement x| & \verb|~| x    & Bitwise complement \\
\verb|shiftL x n|   & \verb|x << n| & Left shift  \\
\verb|shiftR x n|   & \verb|x >> n| & Right shift (arith if x is signed, logical otherwise) \\
\hline
\end{tabular}

% ****************************************************************

\section{File Arch\_Defs.hs: basic architectural definitions}

\markboth{Basic Architectural Definitions}{\copyrightnotice}

\label{sec_arch_defs}

% ================================================================

\subsection{Base ISA type}

The following defines a data type {\tt RV} with two possible values,
{\tt RV32} and {\tt RV64}.  It is analogous to an ``enum'' declaration
in C, defining a family of constants.  The {\tt deriving} clause says
that Haskell can automatically extend the equality operator {\tt ==}
to work on values of type {\tt RV}, and that Haskell can automatically
extend the {\tt show()} function to work on such values, producing
printable Strings {\tt "RV32"} and {\tt "RV64"}, respectively.

\input{Extracted/RV.tex}

% ================================================================

\subsection{Key architectural types: instructions and registers}

Througout the spec, we use Haskell's ``unbounded integer'' type
(\verb'Integer') to represent values that are typically represented in
bit vectors of fixed size in hardware.  Unbounded integers are truly
unbounded and have no limit such as the typical 32 bits or 64 bits
found in most programming languages.  Unbounded integers never
overflow. In the spec, we take care of 32-bit and 64-bit overflow
explicitly.

Below, we define Haskell ``type synonyms'' as more readable synonyms
for Haskell's {\tt Integer} type.

\input{Extracted/Instr.tex}

Next, we have Haskell function that decides whether a particular
instruction is a 32-bit instruction or a 16-bit (compressed)
instruction, by testing its two least-significant bits.

\input{Extracted/is_instr_C.tex}

Here, and everywhere in the spec, you can safely ignore the
\verb|INLINE| annotation.  These are ``pragmas`` or ``directives'' to
the Haskell compiler when we compiler this spec into a simulator, and
are purely meant to improve the performance (speed) of the simulator.
In accordance with the their unimportance to the semantics, we always
write these \verb|INLINE| annotations \emph{below} the corresponding
function.

% ================================================================

\subsection{Major Opcodes for 32-bit instructions}

The 7 least-significant bits of a 32-bit instruction constitute its
``major opcode''.  This section defines them for the base ``I''
instruction set.

\input{Extracted/Major_Opcodes.tex}

Later in the file, we see major-opcode definitions for the ``A''
(Atomics) extension\footnote{I am grateful to my colleague Joe Stoy
for introducing me to the etymology of the word ``atomic''.  It can be
read as ``a+tom+ic''.  The ``tom'' (as in ``tomography'') means
``cut'', and the ``a`` negates it ($\rightarrow$ ``uncuttable'').}, the ``F'' and
``D'' extensions (single-precision and double floating point).

Most instructions also have other fields that further refine the
opcode; we call them ``sub-opcodes''.  These are generally defined in
the separate modules for each extention (for example, in a section
labelled ``Sub-opcodes for 'I' instructions'' in file
\verb|Forvis_Spec_I.hs|) because they are only used locally there.

However, some sub-opcodes are used in multiple modules and are
therefore defined in this file (\verb|Arch_Defs.hs|).  This is true of
all the memory-operation sub-opcodes such as \verb|funct3_LB|,
\verb|funct3_SB|, \verb|msbs5_AMO_ADD|, etc., as well as
\verb|funct3_PRIV|.

% ================================================================

\subsection{Exception Codes}

We define a type synonym for exception codes, and the values of all
the standard exception codes for interrupts:

\input{Extracted/exception_codes_A}

... and for traps:

\input{Extracted/exception_codes_B}

% ================================================================

\subsection{Memory responses}

We define a type {\tt Mem\_Result} for responses from memory.  This
may be {\tt Mem\_Result\_Ok} (successful), in which case it returns a
value (irrelevant for STORE instructions, but relevant for LOAD,
load-reserved, store-conditional, and AMO ops).  Otherwise it is a
{\tt Mem\_Result\_Err}, in which case it returns an exception code
(such as misalignment error, an access error, or a page fault.)

\input{Extracted/Mem_Result}

When returning a result, we construct expressions like these:
\begin{tabbing}
\hmmm \= {\tt Mem\_Result\_Ok} \hm \= \emph{value-expression} \\
      \> {\tt Mem\_Result\_Err}    \> \emph{exception-value-expression}
\end{tabbing}

When fielding a result, we deconstruct it using a case-expression like this:
\begin{tabbing}
\hmmm \= {\tt case} mem-result {\tt of} \\
      \> \hm \= {\tt Mem\_Result\_Ok v} \hm {\tt ->} \= \emph{use} v \emph{in an expression} \\
      \>     \> {\tt Mem\_Result\_Err ec}   {\tt ->} \> \emph{use} ec \emph{in an expression}
\end{tabbing}

% ================================================================

\subsection{Privilege Levels}

RISC-V defines 3 standard privilege levels: Machine, Supervisor and User:

\input{Extracted/Priv_Level}

% ****************************************************************

\section{Files GPR\_File.hs (General Purpose Registers)}

\label{sec_gprs}

\verb|GPR_File.hs| implements a file of general-purpose registers.

We represent it using Haskell's \verb|Data_Map.Map| type, which is an
associative map (like a Python ``dictionary'') that associates
register names with values).  This representation choice is purely
internal to this module because in the export list in the module
header at the top of the file:

\input{Extracted/GPR_File_header}

we mention the type \verb|GPR_File| without exporting its internal
representation.  If, instead, we had said ``\verb|GPR_File(..)|'',
we'd expose its internal detail.  Thus, we can freely change the
representation to something else (and change the API functions
accordingnly) without affecting any of the rest of the modules.  In
other words, \verb|GPR_File| is an \emph{abstract type} for the rest
of the modules.  Here is the representation and constructor:

\input{Extracted/GPR_File}

The \verb|zip| function constructs a listof intial values, associating
each register name 0..31 with 0 (randomly chosen, since the spec does
not specify the initial value of any register).

This is followed by the API functions \verb|gpr_read| and
\verb|gpr_write|.  The latter always writes 0 into GPR 0, so we can
only ever read 0 from GPR 0.\footnote{In ``{\tt seq~val1~(..)}'' in
the last line of {\tt gpr\_write}, only the part in parentheses is
relevant, doing the actual GPR register file update; the rest is a
wrapper that is merely a Haskell performance optimization for the
simulator, concerned with Haskell's lazy evaluation regime.}

% ****************************************************************

\section{File Machine\_State.hs: architectural and machine state}

\markboth{Architectural and Machine State}{\copyrightnotice}

\label{sec_machine_state}

[Reminder: this is for the simple, sequential,
one-instruction-at-a-time interpreter.  The concurrent interpreter has
a substantially different machine state.]

% ================================================================

\subsection{Handling RV32 and RV64 simultaneously}

Although each hardware implementation will typically be either an RV32
system or an RV64 system, the spec encompasses implementations that
can simultaneously support both.  For example, machine-privilege code
may run in RV64 mode while supervisor- and user-privilege code may run
in RV32 mode.  There is also a future RV128 being defined.

In Forvis, which covers RV32 and RV64 and their simultaneous use, we
represent everything using \emph{unbounded} integers (Haskell's
``\verb|Integer|'' type).  The semantics of each instruction are
defined to be governed by the current RV setting which is available in
the architectural state (specifically, MISA.MXL, MSTATUS.SXL,
MSTATUS.UXL, etc.).  An RV32 setting can render some instructions
illegal, and limits calculations on values to 32-bit arithmetic.

% ================================================================

\subsection{Machine State}

We define a new type representing a complete ``machine state'', which
is just a record or struct.  The first few fields represent a RISC-V
hart's basic architectural state: a Program Counter, general purpose
registers, floating-point regsiters, control-and-status Registers, and
the current privilege level at which it is running. This is followed
by two fields representing memory and memory-mapped I/O devices.

Finally, we have fields that are not semantically relevant, but are
needed or useful in simulation or formal reasoning, gathering
statistics, etc., including a list of legal address ranges (memory
load/store instructions should trap if accessing anything outside this
range).

\input{Extracted/Machine_State}

This record-with-fields is a purely internal representation choice in
this module.  Clients of this module only access it via the
\verb|mstate_|{\it{}function} API that follows.\footnote{Haskell has
export-import mechanisms to enforce this external invisibility of our
representation choice, but we have omitted them here to avoid
clutter.}

The following function is a constructor that returns a new machine state:

\input{Extracted/Machine_State_constructor}

All functions that ``update'' the machine state are written in purely
functional style: the first argument is typically a machine state, and
the final result is the new machine state.  This will be evident in
their type signatures:

\begin{tabbing}\tt
\hmm {\it somefunction} :: Machine\_State -> {\it ...other arguments...} -> Machine\_State
\end{tabbing}

What follows is a series of API functions to read or update the
machine state, such as the following to access and update the PC:

\input{Extracted/PC_access}

The \verb|mstate_pc_write| function just applies the \verb|f_pc| field
selector to the machine state to extract that field.  The
\verb|mstate_pc_write| function uses Haskell's ``field update'' notation:

\begin{tabbing}\tt
\hmm \verb|mstate { f_pc = val }|
\end{tabbing}
to construct (and return) a new machine state in which the \verb|f_pc|
field has the new value.

In many of the API functions, such as those to read and write GPRs,
FPRs or CSRs, the function merely invokes the appropriate API of the
corresponding component (GPR file, FPR file or CSR file).

In the API functions for read, write and atomic memory operations,
such as \verb|mstate_mem_read|, we check if the given address is a
supported memory address and return an exception if not.  Otherwise,
we triage the address to determine if it is for actual memory or for a
memory-mapped I/O device, and direct the request to the appropriate
component.

The API functions for FENCE, FENCE.I and SFENCE.VMA are currenty
no-ops in the spec since they only come into play when there is
concurrency involving multiple paths to memory from one or more harts
(hardware threads).  The current specs is interpreted in a completely
sequential, one-instruction-at-a-time manner; there is only one path
to memory involving instructions and data; and there is only one hart.

The file ends with a number of functions to aid in simulation, to move
console input and output between the machine state and the console, to
``tick'' IO devices (which logically run concurrently with the CPU), etc.

% ****************************************************************

\section{File Forvis\_Spec.hs: the ISA spec}

\markboth{The ISA Spec}{\copyrightnotice}

\label{sec_ISA_spec}

The entire spec is essentially in this one file.  The major sections
are:
\begin{itemize}

\item A function \verb|instr_fetch| expressing instruction-fetch,
    returning a regular 32-bit instruction, or 'C' compressed 16-bit
    instruction, or an instruction-fetch fault.

\item A large number of functions named \verb|spec_|{\it{}OPCODE}
describing the semantics of all RISC-V instructions.

\item A small number of functions name \verb|finish_|{\it{}scheme}
capturing the few common schemes by which instructions finish (write
Rd, increment PC, increment MINSTRET, trap, ...)

\item A function \verb|mstate_upd_on_trap| (which is perhaps the most
intricate) that updates the machine state for a trap.  It computes the
new privilege level, new PC, new MSTATUS, new MEPC/{\linebreak[0]}SEPC/{\linebreak[0]}UEPC, new
MCAUSE/{\linebreak[0]}SCAUSE/{\linebreak[0]}UCAUSE, and new MTVAL/{\linebreak[0]}STVAL/{\linebreak[0]}UTVAL based on whether
it was an interrupt or synchronous trap, the current privilege level,
the MSTATUS register, MIP and MIE registers, MIDELEG and MEDELEG
registers, MTVEC/{\linebreak[0]}STVEC/{\linebreak[0]}UTVEC

\item A function \verb|exec_instr| (and it's counterpart
\verb|exec_instr_C| for 'C' compressed instructions) that uses all the
\verb|spec_|{\it{}OPCODE} to update the machine state by executing
exactly one instruction.

\item A function \verb|take_interrupt_if_any| that checks the machine
state to see if an interrupt is pending and updates the machine state
if so (to be poised at the trap vector).

\end{itemize}

% ================================================================

\subsection{Instruction fetch}

The start of the code for instruction fetch looks like this:

\input{Extracted/instr_fetch}

The \verb|instr_fetch| function takes the current machine state as
argument, and attempts to read and instruction from memory, returning
a 2-tuple: a \verb|Fetch Result| and the updated machine state.

The \verb|Fetch_Result| can indicate that there was a fault (such as a
memory access fault or page fault) during the attempted memory-read,
or that the instruction is a 16-bit instruction from the 'C' ISA
extension, or that the instructionis a regular 32-bit instruction.  In
all cases, there could have been a change in the machine state, and
hence it returns the updated machine state.

The body of the \verb|instr_fetch| function first checks the 'C' flag
in CSR MISA to see if compressed instructions are supported.  If not,
it reads a 4-byte (32-bit) instruction from memory.  If 'C' is
supported, it first reads 2 bytes from memory, and checks if it
encodes a possible 'C' instruction. If not, it then reads 2 more bytes
from memory and returns it as a full 32-bit instruction.  Of course,
either of these two reads can fault, and this is the reason we read
two bytes at a time: the first read may succeed with a 'C'
instruction, in which case we do not want to encounter a fault for
reading two more bytes which may be unnecessary in the program flow.

% ================================================================

\subsection{General structure of spec instruction-semantics functions}

The spec is written as a collection of functions, generally one per
major opcode (7 least-significant bits of an instruction).  Each
function has the following structure:

{\small
\begin{Verbatim}[frame=single, numbers=left, commandchars=\\\{\}]
spec_{\it{}OPCODE} :: Machine_State -> Instr -> (Bool, Machine_State) \\
spec_{\it{}OPCODE} mstate  instr =
    -- Instr fields: {\it X}-type
    {\it ... extract instruction fields from standard X format ...}

    -- Decode check
    is_legal = {\it ... check if legal OPCODE and fields ...}

    -- Semantics
    mstate1 = {\it ... instruction-specific semantics based on {\tt mstate} ...}

    mstate2 = {\it ... small family of ``finish'' functions}
  in
    (is_legal, mstate2)
\end{Verbatim}
}

The function takes a machine state and an instruction as arguments,
and returns a 2-tuple: a Boolean and a new machine state.  The Boolean
result indicates whether the instruction is a legal OPCODE instruction
(it's 7 least-significant bits code for OPCODE, and other constraints
on other fields are met, such as must-be-zero).

If the Boolean result is True, the second component of the 2-tuple
result is the updated state due to execution of the instruction.

\subsubsection{Example: the ADD instruction}

The ADD instruction is handled by the following function:

\input{Extracted/exec_ADD_1}

This just dispatches to the function \verb|exec_OP| which is used by
all the \verb|opcode_OP| functions.  To this common function, we pass
the Haskell function \verb|alu_add| to indicate the specific function
to be performed on the operands.

The \verb|exec_OP| function is simple.  We read the Rs1 and Rs2
registers, perform the specified \verb|alu_op|, and finish by updating
the destination register \verb|rd| and incrementing the PC (by 4 or 2,
respectively, depending boolean \verb|is_C|, which indicates whether
the current instruction is a regular 32-bit instruction or a 16-bit C
(compressed) instruction.  The \verb|finish| function will also
increment MINSTRET (see discussion in
Sec.~\ref{sec_standard_finish_functions}).

\input{Extracted/exec_ADD_2}

% ================================================================

\subsection{Standard ``finish'' functions}

\label{sec_standard_finish_functions}

All instructions ``finish'' in one of a few common ways, and these are
captured as functions.  For example: this function captures the common
finish of all ALU instructions, which:
\begin{tightlist}

\item write a result value \verb|rd_val| to the GPR \verb|rd|;

\item increment the PC by 4 or 2, depending on boolean \verb|is_C|,
which indicates whether the current instruction is a regular 32-bit
instruction or a 16-bit C (compressed) instruction;

\item and increment the MINSTRET (instructions retired) counter.

\end{tightlist}

\input{Extracted/finish_rd_and_pc_incr}

% ================================================================

\subsection{Interrupts}

\label{sec_interrupts}

\input{Extracted/take_interrupt}

The \verb|take_interrupt_if_any| function can be applied between any
two instruction executions. It uses the function
\verb|fn_interrupt_pending| that examines MSTATUS, MIP, MIE and the
current privilege level to check if there is an interrupt is pending
and the hart is ready to handle it.  If so, it applies
\verb|mstate_upd_on_trap| to update the machine state, which it
returns along with True.  Otherwise, it returns False and the
unchanged machine state.

% ================================================================

\subsection{Sequential (one-instruction-at-a-time)  interpretation}

The sequential interpreter has a machine state \verb|M| as described
in Sec.~\ref{sec_machine_state}, and a list \emph{spec\_fns} of spec
functions as described in the previous section, i.e., each having the
type:

\hmmm {\tt Machine\_State -> Instr -> (Bool, Machine\_State)}

The interpreter performs the following, forever:

\hmm \begin{minipage}[t]{5in}

It uses the memory-access API function \verb|mstate_mem_read| to read
an instruction from \verb|M|.  It then applies each function from
\emph{spec\_fns}, one by one until one of them returns
\verb|(True,M')|, i.e., one of them successfully decodes and executes
the instruction.

\vspace*{1ex}

If all the functions in \emph{spec\_fns} return \verb|(False,...)|,
the interpreter applies the \verb|finish_trap| function to \verb|M|
with the \verb|ILLEGAL_INSTRUCTION| exception code to produce the next
state \verb|M'|.

\end{minipage}

% ****************************************************************

\section{CSR\_File.hs (Control and Status Registers)}

\label{sec_csrs}

\verb|CSR_File.hs| implements a file of Control and Status registers.
The module begins with definitions of reset values for CSRs at the
User, Supervisor and Machine levels of privilege.

% ================================================================

\subsection{CSR addresses}

The next few sections define the addresses of all the standard CSRs
(Control and Status Registers), at User, Supervisor and Machine
Privilege levels.

\input{Extracted/CSR_Addresses}

% ================================================================

\subsection{The MISA CSR}

A key CSR is MISA (``Machine ISA Register'').  The 2 most-significant
bits are called \verb|MXL| and it encodes the current ``native'' width
of the ISA (width of PC and GPRs), which can be 32, 64 or 128 bits.

\input{Extracted/MISA_fields_A}

The lower 26 bits are named by letters of the alphabet, A-Z, with bit
0 being A and Bit 25 begin Z.  The function \verb|misa_flag|, when
given the value in the MISA register and a letter of the alphabet
(uppercase or lowercase), returns a boolean indicating whether the
corresponding bit is set or not.

\input{Extracted/MISA_fields_B}

This is followed by symbolic-name definitions for the integer bit
positions of the 26 alphabets and the MXL field.

\input{Extracted/MISA_fields_C}

% ================================================================

\subsection{The MSTATUS CSR}

Another key CSR is MSTATUS (``Machine Mode Status'').  The code starts
with definitions for the integer bit positions of its fields.

This is followed by two help-functions that are used in defining the
semantics of exceptions (interrupts and traps) and
returns-from-exceptions.  The least-significant 8 fields of MSTATUS
represent a shallow ``stack'' of interrupt-enable and privilege
bits. On an exception, we push new values on to this stack, and when
we return-from-exception we pop the stack. The function
\verb|mstatus_stack_fields| extracts the stack, returning the fields
as an 8-tuple: \verb|(mpp,spp,mpie,spie,upie,mie,sie,uie)|.  The
inverse function, \verb|mstatus_upd_stack_fields| takes an MSTATUS
value and an 8-tuple of new stack values, and returns a new MSTATUS
with the stack updated.

Not all fields in MSTATUS are used (they may be defined in future
versions of the spec).  The next few definitions describe ``masks''
that restrict an MSTATUS value to defined fields so that we do not
disturb the undefined fields.  Some of the fields of MSTATUS are
visible as the SSTATUS (Supervisor Status) and USTATUS (User Status)
at lower privilege levels.  Masks for these views are also defined
here.

The API functions \verb|csr_read| and \verb|csr_write|.  The main
subtlety here is that the certain distinct CSR addresses refer to
``views'' of the same underlying register with various restrictions:

\begin{itemize}
\item
\verb|USTATUS| and \verb|SSTATUS| are restricted views of \verb|MSTATUS|

\item
\verb|UIE| and \verb|SIE| are restricted views of \verb|MIE|

\item
\verb|UIP| and \verb|SIP| are restricted views of \verb|MIP|
\end{itemize}


The functions \verb|mstatus_stack_fields| and
\verb|mstatus_upd_stack_fields| encapsulate reading and writing the
``stack'' in the MSTATUS register containing the ``previous
privilege'', ``previous interrupt enable'' and ``interrupt enable''
fields.  This stack is pushed on traps/interrupts, and popped on
URET/SRET/MRET instructions.

The function \verb|fn_interrupt_pending| was mentioned earlier in
Sec.~\ref{sec_interrupts}; it analyzes the MSTATUS, MIP, MIE and
current privilege level to decide whether a machine/supervisor/user
external/software/timer interrupt is pending, and if so, which one.

% ****************************************************************

\section{File Virtual\_Mem.hs: Virtual Memory}

\label{sec_vm}

Essentially all the code to support virtual memory is in the file
\verb|Virtual_Mem.hs|.

There are four broad classes of memory access: instruction fetch,
loads, stores, and AMOs.  The function \verb|fn_vm_is_active| checks
whether the effective address computed in each kind of memory access
is a virtual memory address that needs to be translated to a physical
memory address.  It examines the current privilege level, and the
value in the ``mode'' field of CSR SATP.  It also takes into account
that if MSTATUS.MPRV is set, then loads, stores and AMOs should be
regarded as occurring at the privilege level MSTATUS.MPP instead of
the current privilege.

\input{Extracted/fn_vm_is_active}

In file \verb|Forvis_Spec.hs|, in the spec functions for the four
classes of memory access (\verb|instr_fetch|, \verb|spec_LOAD|,
\verb|spec_STORE| and \verb|spec_AMO|), the code first invokes
\verb|fn_vm_is_active| to check if virtual-to-physical address
translation is required.  If so, it then invokes the function
\verb|vm_translate| to perform the translation.  This function can
return a memory access fault or page fault, or a successful
translation with a physical address.  We use the \verb|Memory_Result|
type to return this range of results.  The \verb|vm_translate|
function may also modify machine state (``access'' and ``dirty'' bits
in page tables, internal cache- and TLB-tracking state, etc.), and so
the machine state is both an argument and a result of the function.

\input{Extracted/vm_translate}

% ****************************************************************

\section{Other source code files}

\label{sec_misc}

\verb|Bit_Manipulation.hs| contains utilities for bit manipulation,
including sign- and zero-extension, truncation, conversion, etc. that
are relevant for these semantics.

Most of the remaining files are not part of ISA semantics, but
infrastructure for building a ``system'': a boot ROM, a memory, and
I/O devices such as a timer (MTIME and MTIMECMP), a software-interrupt
location (MSIP), and a UART for console I/O.

\verb|Main.hs| is a driver program that just dispatches to one of two
use-cases, \verb|Main_RunProgram.hs| (free-running) or
\verb|Main_TandemVerifier.hs| (Tandem Verification).

\verb|Main_RunProgram.hs| reads RISC-V binaries (ELF), initializes
architecture state and memory, and calls \verb|RunProgram| to run the
loaded program, up to a specified maximum number of instructions.

\verb|Run_Program.hs| contains the FETCH-EXECUTE loop, along with some
heuristic stopping-conditions (maximum instruction count, detected
self-loop, detected non-zero write into \verb|tohost| memory location,
etc.

\verb|Main_TandemVerifier.hs| sets up Forvis to be a slave processs to
a tandem verifier, receiving commands on stdin and sending responses
on stdout.  The commands allow a tandem verifier to initialize
architecture state, execute 1 or more instructions, and query
archtectural state. Responses include tandem verification packets
which the verifier can use to check an implementation.

\verb|Addres_Map.hs| specifies the address map for the ``system'': the
base address and address range for each memory and I/O device.

\verb|Memory.hs| implements a memory model with read, write and AMO
functions.

\verb|MMIO.hs| implements the memory-mapped I/O system.

\verb|Mem_Ops.hs| defines instruction field values that specify the
type and size of memory operations.  These are duplicates of defs in
\verb|Forvis_Spec.hs| where they are in the specs of LOAD, STORE and
AMO instructions.  They are repeated here because this information is
also needed in \verb|Memory.hs|, \verb|MMIO.hs| and other places.

\verb|UART.hs| is a model of the popular National Semiconductor
NS16550A UART.

\verb|Elf.hs| and \verb|Read_Hex_File.hs| are functions for reading
ELF files and ``Hex Memory'' files, respectively.

% ****************************************************************

\newpage

\markboth{BIBLIOGRAPHY}{\copyrightnotice}

\addcontentsline{toc}{section}{Bibliography}

\bibliographystyle{abbrv}
\bibliography{forvis_reading_guide}

% ****************************************************************

\end{document}
